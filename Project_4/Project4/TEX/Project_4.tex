\documentclass{article}
\usepackage{multicol}
\usepackage{graphicx}% Include figure files
\usepackage{dcolumn}% Align table columns on decimal point
\usepackage{bm}% bold math
\usepackage{hyperref}% add hypertext capabilities
\usepackage{booktabs}
\usepackage{listings}
\usepackage{mathtools}
\usepackage{amsmath}
\renewcommand{\abstractname}{\vspace{-\baselineskip}}
\bibliographystyle{plain}
\usepackage[utf8]{inputenc}
\usepackage{verbatim} %for å inkludere filer med tegn LaTeX ikke liker
\usepackage{mathpazo}
\usepackage{float}
\usepackage{algpseudocode}
\newcommand\numberthis{\addtocounter{equation}{1}\tag{\theequation}}
\usepackage[left=20mm,right=20mm,top=33.95mm,bottom=33.95mm]{geometry} 
% Justerer bredden på columns.
\setlength{\columnsep}{1cm}

\begin{document}

\title{Diffusion}
\author{Sebastian Amundsen, Marcus Berget and Andreas Wetzel}

\maketitle

\begin{abstract}

\end{abstract}

\begin{multicols}{2}

\section{Introduction}



\section{Theory}

\subsection{Heat equation}

The general heat equation can be written as,
\begin{equation}
	\frac{\partial T(\textbf{x}, t)}{\partial t} = \frac{k}{C\rho}\nabla^2 T(\textbf{x}, t). \label{eq:gen_heat}
\end{equation}
where $\textbf{x}$ is the spatial vector, $t$ is time, $c_p$ is the specific heat capacity, $\rho$ is the density and $k$ is the thermal conductivity. We can then gather all the constants in the diffusion constant, $D=\frac{k}{C\rho}$. For the first part of the project we will just set the diffusion constant equal to one. The heat equation in one dimension then becomes, 
\begin{equation}
	\frac{\partial T(x,t)}{\partial t} = \frac{\partial^2 T(x,t)}{\partial x^2}, \label{eq:heat_one}
\end{equation}
or 
\begin{equation}
	T_{xx}=T_t.
\end{equation}
\subsection{Numerical methods for solving the heat equation}
We set the initial conditions of equation \eqref{eq:heat_one} at $t=0$ to,
\begin{align}
	T(x,0)=0, \quad 0<x<L
\end{align}
where $L=1$ is the length of the x-region of interest. We set the boundary conditions to
\begin{align}
	T(0, t)=0, \quad t\geq 0,
\end{align}
and
\begin{align}
	T(L, t)= 1, \quad t\geq 0.
\end{align}
Equation \eqref{eq:heat_one} with the mentioned initial conditions and boundary conditions can be solved numerically using the forward Euler method, the backward Euler method and the implicit Crank-Nicholson scheme. 


\section{Method}


\section{Implementation}



\section{Results}


\section{Discussion}



\section{Concluding remarks}




\end{multicols}

\clearpage

\appendix % Her kommer appendix.

\section{Appendix}



\bibliography{References} % Kilder.
\begin{thebibliography}{9}
\bibitem{94}
	Jensen, M.H., 2015, Computational Physics Lecture Notes Fall 2015
\bibitem{95}
	Jensen, M.H., 2017, Computational Physics Lectures: Statistical physics and the Ising Model
\end{thebibliography}

\end{document}