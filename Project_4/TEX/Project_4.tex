\documentclass{article}
\usepackage{multicol}
\usepackage{graphicx}% Include figure files
\usepackage{dcolumn}% Align table columns on decimal point
\usepackage{bm}% bold math
\usepackage{hyperref}% add hypertext capabilities
\usepackage{booktabs}
\usepackage{listings}
\usepackage{mathtools}
\usepackage{amsmath}
\renewcommand{\abstractname}{\vspace{-\baselineskip}}
\bibliographystyle{plain}
\usepackage[utf8]{inputenc}
\usepackage{verbatim} %for å inkludere filer med tegn LaTeX ikke liker
\usepackage{mathpazo}
\usepackage{float}
\usepackage{algpseudocode}
\newcommand\numberthis{\addtocounter{equation}{1}\tag{\theequation}}
\usepackage[left=20mm,right=20mm,top=33.95mm,bottom=33.95mm]{geometry} 
% Justerer bredden på columns.
\setlength{\columnsep}{1cm}

\begin{document}

\title{The ising model}
\author{Sebastian Amundsen, Marcus Berget and Andreas Wetzel}

\maketitle

\begin{abstract}

\end{abstract}

\begin{multicols}{2}

\section{Introduction}

\section{Method}

\subsection*{The analytical expression}

We have the normalization constant Z which defines the partition function:

\begin{equation}
Z(\beta) = \sum_{s} e^{(-\beta E_s)}
\label{eq:Z}
\end{equation}

With $\beta=1/T$, where T is temperature. We can use this partition function to find the probability $P_s$ of finding a system in a state s:

\begin{equation}
P_s=\frac{e^{-(\beta E_s)}}{Z}
\label{eq:P_s}
\end{equation}

Where $E_s$ is the energy in a given state. We have that the mean energy $E_m$ is given by:

\begin{equation}
E_m = \sum_s \frac{E_s e^{-(\beta*E_s)}}{Z}
\label{eq:E_m}
\end{equation}

We have that the mean magnetization $\langle M \rangle$ is given by:

\begin{equation}
\langle M \rangle=\sum_s^M M_s P_s(\beta)=\frac{1}{Z}\sum_s^M M_s e^{-(\beta E_s)}
\label{eq:mM}
\end{equation}

Where $M_s$ is the different magnetizations. 

\subsection{Metropolis algorithm}

The metropolis algorithm only considers ratios between probabilities, which means that we do not need to calculate the partition function at all when we are using the algorithm. 

\section{Results}

\begin{table}[H]
\begin{center}
\caption{Energy and magnetization given number of up spins.}
\begin{tabular}{  |c|c|c|c|c|c| } \hline
$N_{\text{spins up}}$&Degeneracy&E&M \\ \hline
4&1&-8 J&4\\ \hline
3&4&0 &2 \\ \hline
2&4&0&0\\ \hline
2&2&8 J&0\\ \hline
1&4&0&-2\\ \hline
0&1&-8 J&-4\\ \hline
\end{tabular}
\label{tab:up_spins}
\end{center}
\end{table}

\section{Discussion}

\section{Concluding remarks}

\end{multicols}

\clearpage
\appendix \section{Calculations} % Her kommer appendix.

We find the mean magnetization by using equation \ref{eq:mM}:

\begin{equation} \label{eq1}
\begin{split}
\langle M \rangle& = 4 \times P(-8J) + 4 \times 2 \times P(0) + 4 \times 0 \\
& + 2\times0+4 \times 2 \times P(0) + 4 \times 1 \times P(-8)
\end{split}
\end{equation}



\bibliography{References} % Kilder.
\begin{thebibliography}{9}
\bibitem{94}
	kilder
\end{thebibliography}

\end{document}