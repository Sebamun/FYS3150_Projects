\documentclass{article}
\usepackage{multicol}
\usepackage{graphicx}% Include figure files
\usepackage{dcolumn}% Align table columns on decimal point
\usepackage{bm}% bold math
\usepackage{hyperref}% add hypertext capabilities
\usepackage{booktabs}
\usepackage{listings}
\usepackage{mathtools}
\usepackage{amsmath}
\renewcommand{\abstractname}{\vspace{-\baselineskip}}
\bibliographystyle{plain}
\usepackage[utf8]{inputenc}
\usepackage{verbatim} %for å inkludere filer med tegn LaTeX ikke liker
\usepackage{mathpazo}
\usepackage{float}
\usepackage{algpseudocode}
\newcommand\numberthis{\addtocounter{equation}{1}\tag{\theequation}}
\usepackage[left=20mm,right=20mm,top=33.95mm,bottom=33.95mm]{geometry} 
% Justerer bredden på columns.
\setlength{\columnsep}{1cm}

\begin{document}

\title{The ising model}
\author{Sebastian Amundsen, Marcus Berget and Andreas Wetzel}

\maketitle

\begin{abstract}

\end{abstract}

\begin{multicols}{2}

\section{Introduction}

\section{Method}

\subsection*{The analytical expressions}

We have the normalization constant Z which defines the partition function:

\begin{equation}
Z(\beta) = \sum_{s} e^{(-\beta E_s)}
\label{eq:Z}
\end{equation}

With $\beta=1/k_BT$, where T is temperature and $k_B$ is Boltzmann's constant. We can use this partition function to find the probability $P_s$ of finding a system in a state s:

\begin{equation}
P_s=\frac{e^{-(\beta E_s)}}{Z}
\label{eq:P_s}
\end{equation}

Where $E_s$ is the energy in a given state. We have that the mean energy $E_m$ is given by:

\begin{equation}
E_m = \sum_s \frac{E_s e^{-(\beta*E_s)}}{Z}
\label{eq:E_m}
\end{equation}

The mean energy can be used to find the energy variance $\sigma_E^2$:

\begin{equation}
\begin{split}
\sigma_E^2 &= \langle E^2 \rangle - \langle E \rangle^2 \\
&= \sum_s \frac{E_s^2 e^{-(\beta*E_s)}}{Z} - \bigg(\sum_s \frac{E_s e^{-(\beta*E_s)}}{Z}\bigg)^2
\end{split}
\label{eq:E_v}
\end{equation}

This variance can give us the heat capacity $C_v$ of the system:

\begin{equation}
C_v = \frac{1}{kT} \sigma_E^2 
\label{eq:C_v}
\end{equation}

We have that the mean magnetization $\langle M \rangle$ is given by:

\begin{equation}
\langle M \rangle=\sum_s^M M_s P_s(\beta)=\frac{1}{Z}\sum_s^M M_s e^{-(\beta E_s)}
\label{eq:mM}
\end{equation}

Where $M_s$ is the different magnetizations. We have that the corresponding magnetic variance is given by:

\begin{equation}
\begin{split}
\sigma_M^2&=\langle M^2 \rangle-\langle M \rangle^2 \\
& = \frac{1}{Z}\sum_s^M M_s^2 e^{-(\beta E_s)}
\end{split}
\label{eq:M_v}
\end{equation}

We can use the magnetic variance to find the susceptibility $\chi$:

\begin{equation}
\chi = \frac{1}{k_BT}\sigma_M^2
\end{equation}

\subsection*{Specific case for a 2 X 2 lattice}

We can use our analytical expressions in conjunction with some periodic boundary conditions. We assume two spins in each dimension L=2. If we draw up each lattice with the different spin orientations we can find the degeneracy, energy and magnetization for each state. These values can be used to find the analytical expressions with periodic boundary conditions.

\subsection*{Metropolis algorithm}

The metropolis algorithm only considers ratios between probabilities, which means that we do not need to calculate the partition function at all when we are using the algorithm. 

\section{Results}

\begin{table}[H]
\begin{center}
\caption{Energy and magnetization given number of up spins.}
\begin{tabular}{  |c|c|c|c|c|c| } \hline
$N_{\text{spins up}}$&Degeneracy&E&M \\ \hline
4&1&-8 J&4\\ \hline
3&4&0 &2 \\ \hline
2&4&0&0\\ \hline
2&2&8 J&0\\ \hline
1&4&0&-2\\ \hline
0&1&-8 J&-4\\ \hline
\end{tabular}
\label{tab:up_spins}
\end{center}
\end{table}

In Table \ref{tab:up_spins} we have the energy and magnetization given the number of up spins. 

\subsection*{Specific case for a 2 X 2 lattice}

We have that the partition function for our specific $2\times2$ lattice case is given by:

\begin{equation}
Z=4\cosh{(8J\beta)}+12
\label{eq:Z_22}
\end{equation}

The mean energy is given by:

\begin{equation}
E_m=-8\bigg(\frac{\sinh{(8J\beta)}}{\cosh{(8J\beta)}+4}\bigg)
\label{eq:E_22}
\end{equation}

The mean magnetization is given by:

\begin{equation}
\langle M \rangle=0
\label{eq:M_22}
\end{equation}

The calculations are given in appendix X.

\section{Discussion}

\section{Concluding remarks}

\end{multicols}

\clearpage

\appendix \section{Calculations} % Her kommer appendix.

The partition function for the $2\times2$ lattice is given by equation \ref{eq:Z}:

\begin{equation}
\begin{split}
Z&=2e^{8J\beta}+2e^{-8J\beta}+12e^0\\
&=4\cosh{(8J\beta)}+12
\end{split}
\label{eq:calc_Z}
\end{equation}

This expression combined with equation \ref{eq:E_m} gives the energy:

\begin{equation}
\begin{split}
E_m &= \frac{2\times8Je^{-8J\beta}-2\times8Je^{8J\beta}}{Z}\\
&=\frac{-16J(e^{8J\beta}-e^{-8J\beta})}{Z}\\
&=\frac{-32\sinh{(8J\beta)}}{4\cosh{(8J\beta)}+12}\\
&=-8\bigg(\frac{\sinh{(8J\beta)}}{\cosh{(8J\beta)}+3}\bigg)
\label{eq:calc_E}
\end{split}
\end{equation}

We find the mean magnetization by using equation \ref{eq:mM}:

\begin{equation} \label{eq:calc_M}
\begin{split}
\langle M \rangle& = 1\times 4 \times P(-8) + 4 \times 2 \times P(0) + 4 \times 0 \\
& + 2\times0+4 \times (-2) \times P(0) +  1\times(-4) \times P(-8)\\
&=4\bigg(\frac{e^{-8J\beta }}{4\cosh{(8J\beta)}+12}\bigg)-4\bigg(\frac{e^{-8J\beta }}{4\cosh{(8J\beta)}+12}\bigg) \\
&=0
\end{split}
\end{equation}

\bibliography{References} % Kilder.
\begin{thebibliography}{9}
\bibitem{94}
	kilder
\end{thebibliography}

\end{document}