\documentclass[norsk,a4paper,12pt]{article}
\usepackage[utf8]{inputenc}
\usepackage[T1]{fontenc} %for å bruke æøå
\usepackage[utf8]{inputenc}
\usepackage{graphicx} %for å inkludere grafikk
\usepackage{verbatim} %for å inkludere filer med tegn LaTeX ikke liker
\usepackage{mathpazo}
\usepackage{amsmath}
\usepackage{float}
\usepackage{amsmath}
\usepackage{hyperref}
\newcommand\numberthis{\addtocounter{equation}{1}\tag{\theequation}}
\bibliographystyle{plain}

\title{Project 3}
\author{Sebastian Amundsen, Marcus Berget and Andreas Wetzel}
\date{October 2020}

\begin{document}

\maketitle

\section{Method}
\subsection{Forward Euler}
The forward Euler method is a algorithm to estimate the solution of a differential equation. The Forward Euler method wants to find the next point. To find the next point, it uses the point it is at, $r_n$, a small time step, $dt$, and the derivative of its position. Which can be expressed like this
\begin{align}
    y_{n+1}=y_n + y_n'\cdot dt
\end{align}
Where $y_{n+1}$ is the next step, $y_n$ is the current step, $y_n'$ is the derived of the current step and $dt$ is the time step.\\
This algorithm is really based abbreviated version of a Taylor expansion, where we only expand the series one step at a time. But by only taking one step at a time, we will also get a local truncation error, which causes an error for each step we take. \begin{align}
    y(t_n+dt)=y_{n+1}=y(t_n)+y_n'\cdot dt + R(dt^2)
\end{align} 
Where $R(dt^2)$ is the local truncation error. \\
Since the Forward Euler method a first-order method, will the local truncation error be proportional to the square of the step size. 
\\
\\
\subsection{Verlet method}
\begin{align}
    v_{t+dt}=v_t +\frac{dt}{2}(\frac{F_t}{m}+\frac{F_{t+m}}{m}) R(dt^3)
\end{align}
\end{document}
