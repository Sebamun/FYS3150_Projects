\documentclass{article}
\usepackage{multicol}
\usepackage{graphicx}% Include figure files
\usepackage{dcolumn}% Align table columns on decimal point
\usepackage{bm}% bold math
\usepackage{hyperref}% add hypertext capabilities
\usepackage{booktabs}
\usepackage{listings}
\usepackage{mathtools}
\usepackage{amsmath}
\renewcommand{\abstractname}{\vspace{-\baselineskip}}
\begin{document}

\title{Project 3}
\author{Sebastian Amundsen, Marcus Berget and Andreas Wetzel}
\date{\today}

\maketitle

\begin{abstract}
In this report we looked at the material polymer, which is commonly used in shoe soles. We compared the energy that can be stored in a polymer material and the energy which can be stored in a spring. We found an expression for the potential energy in a polymer when a given force is applied to the material. We showed that a spring is better at storing energy than the polymer material. It is however impractical to use springs as material for a shoe sole.
\end{abstract}

\begin{multicols}{2}

\section{Method}

\subsection{Forward Euler}
The forward Euler method is a algorithm to estimate the solution of a differential equation. The Forward Euler method wants to find the next point. To find the next point, it uses the point it is at, $r_n$, a small time step, $dt$, and the derivative of its position. Which can be expressed like this

\begin{equation}
y_{n+1}=y_n + y_n'\cdot dt
\label{eq:yn1}
\end{equation}

Where $y_{n+1}$ is the next step, $y_n$ is the current step, $y_n'$ is the derived of the current step and $dt$ is the time step.\\
This algorithm is really based abbreviated version of a Taylor expansion, where we only expand the series one step at a time. But by only taking one step at a time, we will also get a local truncation error, which causes an error for each step we take. 

\begin{equation}
\begin{split}
&y(t_n+dt)=y_{n+1}\\
&=y(t_n)+y_n'\cdot dt + R(dt^2)
\end{split}
\label{eq:ytndt}
\end{equation} 

Where $R(dt^2)$ is the local truncation error. Since the Forward Euler method is a first-order method, will the local truncation error be proportional to the square of the step size. 

\subsection{Velocity Verlet method}

The velocity Verlet method is based on the kinematic equations for an moving object, which in our case is the earth's orbit around the sun. If we want to find the next time step for the velocity and position we do a approximation and uses Taylor-expansion   

\begin{equation}
    v_{t+dt}=v_t +\frac{dt}{2}(\frac{F_t}{m}+\frac{F_{t+m}}{m}) R(dt^3)
\end{equation}\\

We can also split the equation above and perform the calculation in several steps like this:

\begin{equation}
\begin{split}
&v(t+\frac{1}{2}\Delta t)=v(t)+\frac{1}{2}a(t)\Delta t\\
&x(t+\Delta t)=x(t)+v(t+\frac{1}{2}\Delta t)\Delta t\\
&a(t+\Delta t)=f(x(t+\Delta t))\\
&v(t+\Delta t)=v(t+\frac{1}{2}\Delta t)+\frac{1}{2}a(t+\Delta t)\Delta t
\end{split}
\label{eq:steps}
\end{equation}

\end{multicols}

\end{document}
