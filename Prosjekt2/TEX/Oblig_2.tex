\documentclass[norsk,a4paper,12pt]{article}
\usepackage[utf8]{inputenc}
\usepackage[T1]{fontenc} %for å bruke æøå
\usepackage[utf8]{inputenc}
\usepackage{graphicx} %for å inkludere grafikk
\usepackage{verbatim} %for å inkludere filer med tegn LaTeX ikke liker
\usepackage{mathpazo}
\usepackage{amsmath}
\usepackage{float}
\usepackage{amsmath}
\usepackage{hyperref}
\newcommand\numberthis{\addtocounter{equation}{1}\tag{\theequation}}
\bibliographystyle{plain}

\begin{document}
\title{FYS3150-Project 2}
\author{Marcus Berget, Sebastian Amundsen, Andreas Wetzel}
\date{August 2020}
\maketitle

\begin{abstract}

\end{abstract}

\section{Introduction}

The aim of this project is solving eigenvalue/eigenvector problems using the Jacobi algorithm. Specifically, we will look at solving the Schroedinger's equation with a three-dimensional harmonic oscillator potential. 

\section{Theory}
\subsection{Jac}

\section{Method}

The first step of solving eigenvalue problems numerically, is of course to set up the matrix. In both the case of solving the buckling beam problem and the quantum mechanical problem the matrix is a tridiagonal matrix
We find the analytical eigenvalues by using armadillos functions for diagonalizing a matrix. These values are to be compared with values found using the Jacobi method. 


The general expression for the new matrix elements are:
$$
b_{ii}=a_{ii}, i\neq k, i\neq l \\
$$
$$
b_{ik}=a_{ik}\cos \theta - a_{il}\sin \theta, i\neq k, i\neq l
$$
$$
b_{il}=a_{il}\cos\theta+a_{ik}\sin\theta, i\neq k, i\neq l
$$
$$
b_{kk}=a_{kk} \cos^2\theta - 2a_{kl}\cos\theta \sin\theta + a_{ll}\sin^2\theta
$$
$$
b_{ll}=a_{ll}\cos^2\theta+2a_{kl}\cos\theta \sin\theta + a_{kk}\sin^2\theta
$$
\begin{equation}
b_{kl} = (a_{kk}-a_{ll})\cos\theta\sin\theta+a_{kl}(\cos^2\theta-\sin^2\theta)
 \label{eq:Jacob}
 \end{equation}

\section{Implementation}



\section{Results}

\section{Discussion}


\section{Concluding remarks}

\bibliography{referanser}

\end{document}
